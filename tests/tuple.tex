\documentclass[12pt]{standalone}

\usepackage{tikz}
\usetikzlibrary{positioning}
\makeatletter

%%%%%%%%%%%%%%%%%%%%%%%%%%%%%%%%%%%%%%%%%%%%%%%%%%%%%%%%%%%%%%%%%%%%%%%%%%%%%%%%
% Defining [funprog] family
%%%%%%%%%%%%%%%%%%%%%%%%%%%%%%%%%%%%%%%%%%%%%%%%%%%%%%%%%%%%%%%%%%%%%%%%%%%%%%%%

\pgfkeys{
  /tikz/.cd,
  funprog/.is family,
  funprog/.search also={/tikz},
}

%%%%%%%%%%%%%%%%%%%%%%%%%%%%%%%%%%%%%%%%%%%%%%%%%%%%%%%%%%%%%%%%%%%%%%%%%%%%%%%%
% [use funprog] 
%%%%%%%%%%%%%%%%%%%%%%%%%%%%%%%%%%%%%%%%%%%%%%%%%%%%%%%%%%%%%%%%%%%%%%%%%%%%%%%%

\pgfkeys{
  /tikz/use funprog/.append code={
    \pgfkeys{
      /tikz/.cd,
      list/.style={funprog, list=##1},
      index/.style={funprog, \funprog@datastructure/index=##1}
    }
  },
  /tikz/funprog/data structure/.store in=\funprog@datastructure,
  /tikz/funprog/element/.store in=\funprog@element,
}

%%%%%%%%%%%%%%%%%%%%%%%%%%%%%%%%%%%%%%%%%%%%%%%%%%%%%%%%%%%%%%%%%%%%%%%%%%%%%%%%
% [list] 
%%%%%%%%%%%%%%%%%%%%%%%%%%%%%%%%%%%%%%%%%%%%%%%%%%%%%%%%%%%%%%%%%%%%%%%%%%%%%%%%

\pgfkeys{
  /tikz/funprog/.cd,
  list/.style={
    column sep=-\pgflinewidth,
    data structure=list,
    element=#1,
    inner sep=0pt,
    name=list #1,
    nodes={
      draw,
      inner sep=0pt,
      minimum size=1cm,
      outer sep=.5\pgflinewidth,
    },
    outer sep=-.5\pgflinewidth,
    row sep=-\pgflinewidth,
  },
  list/.default=a,
  list/index/.style={
    name=\funprog@element#1,
    node contents=$\funprog@element_{#1}$,
  }
}

\makeatother


\begin{document}

\begin{tikzpicture}[
    use funprog,
    node distance=0pt,
    every tuple/.style={
        tuple delimiter=braces,
    }
]

\draw [help lines, step=1cm] (-3, -3) grid (3, 3);

\matrix [tuple=a] {
    \node [elements before]; & \node [separator]; &
    \node [index=1]; & \node [separator]; &
    \node [elements between]; & \node [separator]; &
    \node [index=2]; & \node [separator]; &
    \node [elements after]; \\
};

\matrix [tuple=b, below=1 of tuple a] {
    \node [index=1]; & \node [separator]; &
    \node [index=2]; & \node [separator]; &
    \node [elements between]; & \node [separator]; &
    \node [index=i]; & \node [separator]; &
    \node [elements after]; \\
};

\matrix [tuple=c, base element=x, right=1 of tuple a] {
    \node [index=1]; & \node [separator]; &
    \node [index=2]; & \node [separator]; &
    \node [elements between]; & \node [separator]; &
    \node [index=i]; & \node [separator]; &
    \node [elements after]; \\
};

\draw [<-] (a1.north) -- +(0, 1);
\draw [<-] (bi.south) -- +(1, -1);


\end{tikzpicture}

\end{document}